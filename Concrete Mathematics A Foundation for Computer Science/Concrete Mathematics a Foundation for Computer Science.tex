\documentclass{book}
\usepackage{hyperref}
\usepackage{amsmath}  % for mathematical symbols and environments

\title{Concrete Mathematics (A Foundation For Computer Science)}
\author{Jeanphilo Gong}
\date{Sep, 16, 2023}

\begin{document}

\maketitle  % Generates the title page

\tableofcontents  % Generates a table of contents

\chapter{Recurrent Problems}

\section{The Tower of Hanoi}

\subsection*{Mathematical induction}

Mathematical induction is a way to prove some statement about the integer \( n \) is true for all \( n \geq n_0 \). First we prove the statement when \( n \) has its smallest value, \( n_0 \); this is called the basis. Then we prove the statement for \( n > n_0 \), assuming that it has already been proved for all values between \( n_0 \) and \( n - 1 \), inclusive; this is called the induction. Such a proof gives infinitely many results with only a finite amount of work.

\section{Lines in the Plane}

\section{The Josephus Problem}

\section*{Exercises}

\chapter{Sums}

\chapter{Integer Functions}

\chapter{Number Theory}

\chapter{Binomial Coefficients}

\chapter{Special Numbers}

\chapter{Generating Functions}

\chapter{Discrete Probability}

\chapter{Asymptotics}

\end{document}
