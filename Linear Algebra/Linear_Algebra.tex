\documentclass{book}
\usepackage{hyperref}
\usepackage{amsmath}



\begin{document}

\tableofcontents

\title{Linear Algebra}

% Chapter 1
\chapter{Introduction to Vectors}

\section{Vectors and Linear Combinations}

\section{Lengths and Dot Products}

\section{Matrices}

\subsection{Linear Equations}

\subsection{The Inverse Matrix}
this part is so hard

\subsection{Cyclic Differences}





% Chapter 2
\chapter{Solving Linear Equations}
\section{Vectors and Linear Equations}

\section{The Idea of Elimination}
\subsection*{Second lecture}
    \subsubsection{Elimination}
    \subsubsection{Back substitution}
    Here, we need to pay attention to the particularity of the matrix
    \begin{align}
        \begin{bmatrix}
          a & b \\
          c & d
        \end{bmatrix}
        \times
        \begin{bmatrix}
          e & f \\
          g & h
        \end{bmatrix}
        =
        \begin{bmatrix}
          ae + bg & af + bh \\
          ce + dg & cf + dh
        \end{bmatrix}
      \end{align}



\section{Elimination Using Matrices}

\section{Rules for Matrix Operations}

\section{Inverse Matrices}

\section{Elimination = Factorization: A = LU}

\section{Transposes and Permutations}



% Chapter 3
\chapter{Vector Spaces and Subspaces}
\section{Spaces of Vectors}

\section{The Nullspace of A: Solving Ax = 0}

\section{The Rank and the Row Reduced Form}

\section{The Complete Solution to Ax = b}

\section{Independence, Basis and Dimension}

\section{Dimensions of the Four Subspaces}










% Chapter 4
\chapter{Orthogonality}
\section{Orthogonality of the Four Subspaces}

\section{Projections}

\section{Least Squares Approximations}

\section{Orthogonal Bases and Gram-Schmidt}







% Chapter 5
\chapter{Determinants}
\section{The Properties of Determinants}

\section{Permutations and Cofactors}

\section{Cramer's Rule, Inverse, and Volumes}








% Chapter 6
\chapter{Elgenvalues and Elgenvectors}
\section{Introduction to Eigenvalues}

\section{Diagonalizing a Matrix}

\section{Applications to Differential Equations}

\section{Symmetric Matrices}

\section{Positive Definite Matrices}

\section{Similar Matrices}

\section{Singular Value Decomposition}








% Chapter 7
\chapter{Linear Transformations}
\section{The idea of a Linear Transformation}

\section{The Matrix of a Linear Transformation}

\section{Diagonalization and the Pseudoinverse}









% Chapter 8
\chapter{Applications}
\section{Matrices in Engineering}

\section{Graphs and Networks}

\section{Markov Matrices, Population, and Economics}

\section{Linear Programming}

\section{Fourier Series: Linear Algebra for Functions}

\section{Linear Algebra for Statistics and Probability}

\section{Computer Graphics}










% Chapter 9
\chapter{Numerical Linear Algebra}
\section{Gaussian Elimination in Practice}

\section{Norms and Condition Numbers}

\section{Iterative Methods and Preconditioners}










% Chapter 10
\chapter{Complex Vectors and Matrices}
\section{Complex Numbers}

\section{Hermitian and Unitary Matrices}

\section{The Fast Fourier Transform}




\end{document}
